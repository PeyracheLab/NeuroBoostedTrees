\documentclass[10pt,a4paper,twocolumn]{article}
\usepackage[utf8]{inputenc}
\usepackage[english]{babel}
\usepackage{amsmath}
\usepackage{amsfonts}
\usepackage{amssymb}
\usepackage{graphicx}
\usepackage{float}
\usepackage{hyperref}
\usepackage{layouts}
\usepackage[left=2cm,right=2cm,top=2cm,bottom=2cm]{geometry}
\usepackage{placeins}
\author{Guillaume Viejo}
\title{Reports for Prediction_ML_GLM}

\begin{document}
\onecolumn
Reports for 

\textbf{Modern machine learning far outperforms GLMs at predicting spikes} 

by Benjamin et al, 2017

Benchmarking with Post-subiculum (Pos) and Antero-dorsal nucleus (ADn) recordings. Data can be found in $~/Dropbox/Peyrache\ Lab\ Team\ Folder/Data/HDCellData/data\_test\_boosted\_tree.mat$


Scoring function :
\begin{equation}
	pseudo-R^2 = 1 - \frac{(y \log y - y) - (y \log \hat{y} - \hat{y})}{(y \log y - y) - (y \log \bar{y} - \bar{y})}
\end{equation}
with:
\begin{itemize}
	\item $y$ the target firing rate
	\item $\hat{y}$ the prediction
	\item $\bar{y}$ the mean firing rate
\end{itemize}

Machine learning techniques are :
\begin{itemize}
	\item GLM : \url{https://github.com/glm-tools/pyglmnet}
	\item NN : 
	\item XGB : \url{http://xgboost.readthedocs.io/en/latest/model.html}
\end{itemize}

\tableofcontents

\section{Benchmarking of GLM, NN and XGB}

\twocolumn

\subsection{Features: Cosinus and sinus of the head direction \\ Target : Pos and ADn}


\begin{figure}[H]
	\begin{center}
		\includegraphics[width=0.9\linewidth]{../figures/1_PR2_feat_cos_sin_targ_pos_adn.pdf} 
	\end{center}
\end{figure}

\subsection{Features : Cos, Sin, X position, Y position, Ang (rad), Velocity \\ Target : Pos and ADn}

\begin{figure}[H]
	\begin{center}
		\includegraphics[width=0.9\linewidth]{../figures/2_PR2_feat_all_targ_pos_adn.pdf} 
	\end{center}
\end{figure}

\subsection{Features : X position, Y position, Ang (rad), Velocity \\ Target : Pos and ADn}

\begin{figure}[H]
	\begin{center}
		\includegraphics[width=0.9\linewidth]{../figures/3_PR2_feat_raw_targ_pos_adn.pdf} 
	\end{center}
\end{figure}

\subsection{Features : Pos firing rate \\ Target : ADn firing rate}

\begin{figure}[H]
	\begin{center}
		\includegraphics[width=0.9\linewidth]{../figures/4_PR2_feat_pos_targ_adn.pdf} 
	\end{center}
\end{figure}

\subsection{Features : ADn firing rate \\ Target : POS firing rate}

\begin{figure}[H]
	\begin{center}
		\includegraphics[width=0.9\linewidth]{../figures/5_PR2_feat_adn_targ_pos.pdf} 
	\end{center}
\end{figure}

\subsection{Features : Cos, Sin, Pos \\ Target : ADn firing rate}

\begin{figure}[H]
	\begin{center}
		\includegraphics[width=0.9\linewidth]{../figures/6_PR2_feat_cos_sin_pos_targ_adn.pdf} 
	\end{center}
\end{figure}

\clearpage

\subsection{Features : Cos, Sin, ADn \\ Target : Pos firing rate}

\begin{figure}[H]
	\begin{center}
		\includegraphics[width=0.9\linewidth]{../figures/7_PR2_feat_cos_sin_adn_targ_pos.pdf} 
	\end{center}
\end{figure}

\subsection{Features : All + Pos \\ Target : ADn firing rate}

\begin{figure}[H]
	\begin{center}
		\includegraphics[width=0.9\linewidth]{../figures/8_PR2_feat_all_pos_targ_adn.pdf} 
	\end{center}
\end{figure}

\subsection{Features : All + ADn \\ Target : Pos firing rate}

\begin{figure}[H]
	\begin{center}
		\includegraphics[width=0.9\linewidth]{../figures/9_PR2_feat_all_adn_targ_pos.pdf} 
	\end{center}
\end{figure}

\subsection{Features : Ang \\ Target : Pos and ADn}

Comparison of XGB with the tuning curve of each model based on angular direction 
Tuning curve is made with 100 bins.

\begin{figure}[H]
	\begin{center}
		\includegraphics[width=0.9\linewidth]{../figures/10_PR2_feat_ang_targ_pos_adn.pdf} 
	\end{center}
\end{figure}


\onecolumn

\section{Test of XGBoost}

\subsection{Segmentation of angular direction for 6 neurons}

The blue line is the tuning curve of the neuron.
The vertical grey line are the split value of XGBoost. There are 100 trees with 5 layers for each booster.

\begin{figure}[H]
	\begin{center}
		\includegraphics[width=1\linewidth]{../figures/11_XGB_threshold_angular_x.pdf}
	\end{center}
\end{figure}

\subsection{Segmentation of (x, y) space for 6 neurons}

\begin{figure}[H]
	\begin{center}
		\includegraphics[width=1\linewidth]{../figures/12_XGB_threshold_xy.pdf}
	\end{center}
\end{figure}


\subsection{Segmentation of (x, y) space and angular direction for 6 neurons}

\begin{figure}[H]
	\begin{center}
		\includegraphics[width=1\linewidth]{../figures/13_XGB_threshold_xyang.pdf}
	\end{center}
\end{figure}

\subsection{Density of splits}

For all neurons.
For the first row, only the angular firing rate is taken as a feature
For the others rowss, the x and y position

\begin{figure}[H]
	\begin{center}
		\includegraphics[width=1\linewidth]{../figures/14_XGB_threshold_density.pdf}
	\end{center}
\end{figure}




\section{Peer-prediction with XGBoost}

\subsection{Splits of the firing rates}

for each 6 neurons as previously


\begin{figure}[H]
	\begin{center}
		\includegraphics[width=1\linewidth]{../figures/15_XGB_peer_prediction.pdf}
	\end{center}
\end{figure}

\subsection{Splits count per neurons}

Graph made for fun, \#pointless \#pretty\_useless\_figure

\begin{figure}[H]
	\begin{center}
		\includegraphics[width=0.9\linewidth]{../figures/16_XGB_peer_prediction.pdf}
	\end{center}
\end{figure}

\subsection{Prediction of the other population}

\begin{figure}[H]
	\begin{center}
		\includegraphics[width=1\linewidth]{../figures/17_XGB_cross_peer_prediction.pdf}
	\end{center}
\end{figure}

\begin{figure}[H]
	\begin{center}
		\includegraphics[width=0.9\linewidth]{../figures/18_XGB_cross_peer_prediction.pdf}
	\end{center}
\end{figure}

\section{Test of NN}

\end{document}
